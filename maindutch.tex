\selectlanguage{dutch}

\includepdf{./Titelpagina/titelpaginaNL.pdf}
\setcounter{page}{2}
%\tableofcontents

\section*{Inleiding}
\addcontentsline{toc}{section}{Inleiding}
A BV \citep{aoibsfc} is een exporterende onderneming en verkoopt haar producten in een groot deel van Europa. Een deel van de omzet wordt behaald in het Verenigd Koninkrijk. Deze leveringen worden gefactureerd in ponden (GBP) met een betaaltermijn van een maand. De orders worden gemiddeld een tot twee maanden voor levering geplaatst. A heeft als beleid dat het hieruit voortvloeiende valutarisico wordt afgedekt door middel van valutatermijncontracten op het moment dat de orders worden geplaatst. De voor dit doel afgesloten valutatermijncontracten worden vastgelegd in de administratie. De reële waarde (en kostprijs) van de valutatermijncontracten op het moment van afsluiten is nihil. De transactiekosten voor het afsluiten van de valutatermijncontracten worden verwaarloosbaar geacht en worden bij afsluiten van het contract direct ten laste van de winst-en-verliesrekening gebracht.

De valutatermijncontracten zijn bedoeld om het risico van de variabiliteit van de te ontvangen kasstromen af te dekken. De omvang en looptijd van de valutatermijncontracten komen overeen met de omvang en verwachte betalingstermijnen van de afgesloten verkooporders. Derhalve worden de afgesloten valutatermijncontracten als zeer effectief aangemerkt (dit zal voorts in ieder geval op iedere balansdatum beoordeeld moeten worden).


\glssetwidest{geld-goederenbeweging }
\setglossarystyle{alttree} %longheader, list, super
\printglossary[title=Begrippenkader]
\gls{treasuryletter} \gls{treasury} \gls{geldgoederen} \gls{fx} \gls{valuta} \gls{rente} \gls{koers} \gls{rating} \gls{financiering} \gls{geldstromen} \gls{derivaten}



\stepcounter{artikel}
\subsection*{Artikel \theartikel \hspace{1em} Algemene hedgestrategie}
\addcontentsline{toc}{subsection}{Artikel \theartikel \hspace{1em} Algemene hedgestrategie}
Strategie is om het valutarisico op afgesloten verkooporders in GBP volledig af te dekken. Daartoe worden valutatermijncontracten afgesloten waarbij GBP worden verkocht in ruil voor euro’s tegen een vaste koers.


\stepcounter{artikel}
\subsection*{Artikel \theartikel \hspace{1em} Doelstelling van de treasury letter}
\addcontentsline{toc}{subsection}{Artikel \theartikel \hspace{1em} Doelstelling van de treasury letter}
De doelstelling is om niet bloot te staan aan het risico van de variabiliteit van de te ontvangen kasstromen uit hoofde van verkopen in GBP. Op grond van deze doelstellingen dienen elke periode de afgesloten verkooporders in GBP te worden afgedekt door middel van valutatermijncontracten.

\section*{Risicobeheer}
\stepcounter{artikel}
\subsection*{Artikel \theartikel \hspace{1em} Uitgangspunten risicobeheer}
\addcontentsline{toc}{subsection}{Artikel \theartikel \hspace{1em} Uitgangspunten risicobeheer}


\stepcounter{artikel}
\subsection*{Artikel \theartikel \hspace{1em} FX-risicobeheer}
\addcontentsline{toc}{subsection}{Artikel \theartikel \hspace{1em} FX-risicobeheer}
{\color{red}Hier moet nog komen rente, koers, krediet, interne liquiditeit en valutarisico}



\stepcounter{artikel}
\subsection*{Artikel \theartikel \hspace{1em} Soort hedgerelatie}
\addcontentsline{toc}{subsection}{Artikel \theartikel \hspace{1em} Soort hedgerelatie}
Kasstroomhedge: afdekking van het risico van de variabiliteit van de te ontvangen kasstromen.


\stepcounter{artikel}
\subsection*{Artikel \theartikel \hspace{1em} Toegepast hedge-accountingmodel}
\addcontentsline{toc}{subsection}{Artikel \theartikel \hspace{1em} Toegepast hedge-accountingmodel}
Kostprijshedge-accounting


\stepcounter{artikel}
\subsection*{Artikel \theartikel \hspace{1em} Aard van het afgedekte risico}
\addcontentsline{toc}{subsection}{Artikel \theartikel \hspace{1em} Aard van het afgedekte risico}
Het risico van de variabiliteit van de te betalen kasstromen (valutarisico).


\stepcounter{artikel}
\subsection*{Artikel \theartikel \hspace{1em} Verwachte effectiviteit}
\addcontentsline{toc}{subsection}{Artikel \theartikel \hspace{1em} Verwachte effectiviteit}
De omvang en looptijd van de valutatermijncontracten komen overeen met de omvang en verwachte betalingstermijnen van de afgesloten verkooporders. Derhalve worden de afgesloten valutatermijncontracten als zeer effectief aangemerkt.

\stepcounter{artikel}
\subsection*{Artikel \theartikel \hspace{1em} Identificatie van afgedekte positie}
\addcontentsline{toc}{subsection}{Artikel \theartikel \hspace{1em} Identificatie van afgedekte positie}
De per periode afgesloten verkooporders in GBP worden vastgelegd in de verkoopadministratie, inclusief de verwachte leverdatum en betalingstermijn. Wijzigingen in omvang en/of verwachte leverdatum en betalingstermijn (inclusief annuleringen) worden eveneens vastgelegd. Op deze wijze blijkt uit de verkoopadministratie steeds de per periode te verwachten kasstroom in GBP uit hoofde van afgesloten, maar nog niet uitgeleverde verkooporders. De per periode te verwachten kasstroom in GBP uit hoofde van uitgeleverde verkooporders blijkt uit de debiteurenadministratie.


\stepcounter{artikel}
\subsection*{Artikel \theartikel \hspace{1em} Identificatie van hedginginstrumenten}
\addcontentsline{toc}{subsection}{Artikel \theartikel \hspace{1em} Identificatie van hedginginstrumenten}
De omvang en looptijd van de af te sluiten valutatermijncontracten worden bepaald overeenkomstig de omvang en verwachte betalingstermijnen van de
afgesloten verkooporders. Van de per periode afgesloten valutatermijncontracten worden de volgende gegevens vastgelegd in de administratie:

- afsluitdatum;

- tegenpartij en contractnummer;

- omvang in GBP;

- termijnkoers;

- omvang in EURO;

- datum afloop.


\stepcounter{artikel}
\subsection*{Artikel \theartikel \hspace{1em} Testen van de effectiviteit}
\addcontentsline{toc}{subsection}{Artikel \theartikel \hspace{1em} Testen van de effectiviteit}
De effectiviteit van de hedge zal aan het einde van iedere periode worden beoordeeld door het vergelijken van de kritische kenmerken van de lopende verkooporders en van de ingenomen valutatermijncontracten. Als er sprake is van een overhedge, worden eventuele verliezen als gevolg van deze overhedge direct ten laste van de winst-en-verliesrekening verwerkt.



\newpage
\section*{Administratieve organisatie en interne controle}
\stepcounter{artikel}
\subsection*{Artikel \theartikel \hspace{1em} Uitgangspunten AO/IC}
\addcontentsline{toc}{subsection}{Artikel \theartikel \hspace{1em} Uitgangspunten AO/IC}


\stepcounter{artikel}
\subsection*{Artikel \theartikel \hspace{1em} Verantwoordelijkheden en bevoegdheden}
\addcontentsline{toc}{subsection}{Artikel \theartikel \hspace{1em} Verantwoordelijkheden en bevoegdheden}

\begin{table}[!h]
    \centering
    \caption{Verantwoordelijkheden en bevoegdheden omtrent treasury}
    \begin{tabular}{l l l}
        \toprule
        \textbf{Functie} & \textbf{Verantwoordelijkheden} & \textbf{Bevoegdheden} \\
        \midrule
        De directie & blabla & blabla \\
        Afdelingsmanagers & blabla & blabla \\
        Beschikkende medewerkers & blabla & blabla \\
        \bottomrule
    \end{tabular}
    \label{tab:functiescheiding}
\end{table}


\stepcounter{artikel}
\subsection*{Artikel \theartikel \hspace{1em} Managementinformatie}
\addcontentsline{toc}{subsection}{Artikel \theartikel \hspace{1em} Managementinformatie}


\section*{De werking hiervan}
\stepcounter{artikel}
\subsection*{Artikel \theartikel \hspace{1em} Inwerkingtreding}
\addcontentsline{toc}{subsection}{Artikel \theartikel \hspace{1em} Inwerkingtreding}
Deze treasury letter treedt met terugwerkende kracht in werking vanaf 1 januari 2018 en vervangt voorgaande generieke documentatie op 13 november 2008 is vastgesteld door de Raad van de gemeente Ede.

%\printbibliography
%\addcontentsline{toc}{section}{Referenties}

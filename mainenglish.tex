\selectlanguage{english}

\includepdf{./Titelpagina/titelpaginaEN.pdf}
\setcounter{page}{2}
%\tableofcontents

\section*{Preface}
\addcontentsline{toc}{section}{Preface}
A BV \citep{aoibsfc} is een exporterende onderneming en verkoopt haar producten in een groot deel van Europa. Een deel van de omzet wordt behaald in het Verenigd Koninkrijk. Deze leveringen worden gefactureerd in ponden (GBP) met een betaaltermijn van een maand. De orders worden gemiddeld een tot twee maanden voor levering geplaatst. A heeft als beleid dat het hieruit voortvloeiende valutarisico wordt afgedekt door middel van valutatermijncontracten op het moment dat de orders worden geplaatst. De voor dit doel afgesloten valutatermijncontracten worden vastgelegd in de administratie. De reële waarde (en kostprijs) van de valutatermijncontracten op het moment van afsluiten is nihil. De transactiekosten voor het afsluiten van de valutatermijncontracten worden verwaarloosbaar geacht en worden bij afsluiten van het contract direct ten laste van de winst-en-verliesrekening gebracht.

De valutatermijncontracten zijn bedoeld om het risico van de variabiliteit van de te ontvangen kasstromen af te dekken. De omvang en looptijd van de valutatermijncontracten komen overeen met de omvang en verwachte betalingstermijnen van de afgesloten verkooporders. Derhalve worden de afgesloten valutatermijncontracten als zeer effectief aangemerkt (dit zal voorts in ieder geval op iedere balansdatum beoordeeld moeten worden).


\glssetwidest{geld-goederenbeweging }
\setglossarystyle{alttree} %longheader, list, super
\printglossary[title=Glossary]
\gls{treasuryletter} \gls{treasury} \gls{geldgoederen} \gls{fx} \gls{valuta} \gls{rente} \gls{koers} \gls{rating} \gls{financiering} \gls{geldstromen} \gls{derivaten}



\stepcounter{artikel}
\subsection*{Article \theartikel \hspace{1em} General hedge strategy}
\addcontentsline{toc}{subsection}{Article \theartikel \hspace{1em} General hedge strategy}
Strategie is om het valutarisico op afgesloten verkooporders in GBP volledig af te dekken. Daartoe worden valutatermijncontracten afgesloten waarbij GBP worden verkocht in ruil voor euro’s tegen een vaste koers.


\stepcounter{artikel}
\subsection*{Article \theartikel \hspace{1em} Purpose of the treasury letter}
\addcontentsline{toc}{subsection}{Article \theartikel \hspace{1em} Purpose of the treasury letter}
De doelstelling is om niet bloot te staan aan het risico van de variabiliteit van de te ontvangen kasstromen uit hoofde van verkopen in GBP. Op grond van deze doelstellingen dienen elke periode de afgesloten verkooporders in GBP te worden afgedekt door middel van valutatermijncontracten.

\section*{Risk management}
\stepcounter{artikel}
\subsection*{Article \theartikel \hspace{1em} Principles of risk management}
\addcontentsline{toc}{subsection}{Article \theartikel \hspace{1em} Principles of risk management}


\stepcounter{artikel}
\subsection*{Article \theartikel \hspace{1em} FX-risk management}
\addcontentsline{toc}{subsection}{Article \theartikel \hspace{1em} FX-risk management}
Hier moet nog komen rente, koers, krediet, interne liquiditeit en valutarisico



\stepcounter{artikel}
\subsection*{Article \theartikel \hspace{1em} Hedge relation type}
\addcontentsline{toc}{subsection}{Article \theartikel \hspace{1em} Hedge relation type}
Kasstroomhedge: afdekking van het risico van de variabiliteit van de te ontvangen kasstromen.


\stepcounter{artikel}
\subsection*{Article \theartikel \hspace{1em} Applied hedge accounting model}
\addcontentsline{toc}{subsection}{Article \theartikel \hspace{1em} Applied hedge accounting model}
Kostprijshedge-accounting


\stepcounter{artikel}
\subsection*{Article \theartikel \hspace{1em} Nature of hedged risks}
\addcontentsline{toc}{subsection}{Article \theartikel \hspace{1em} Nature of hedged risks}
Het risico van de variabiliteit van de te betalen kasstromen (valutarisico).


\stepcounter{artikel}
\subsection*{Article \theartikel \hspace{1em} Expected effectiveness}
\addcontentsline{toc}{subsection}{Article \theartikel \hspace{1em} Expected effectiveness}
De omvang en looptijd van de valutatermijncontracten komen overeen met de omvang en verwachte betalingstermijnen van de afgesloten verkooporders. Derhalve worden de afgesloten valutatermijncontracten als zeer effectief aangemerkt.

\stepcounter{artikel}
\subsection*{Article \theartikel \hspace{1em} Identification of hedged positions}
\addcontentsline{toc}{subsection}{Article \theartikel \hspace{1em} Identification of hedged positions}
De per periode afgesloten verkooporders in GBP worden vastgelegd in de verkoopadministratie, inclusief de verwachte leverdatum en betalingstermijn. Wijzigingen in omvang en/of verwachte leverdatum en betalingstermijn (inclusief annuleringen) worden eveneens vastgelegd. Op deze wijze blijkt uit de verkoopadministratie steeds de per periode te verwachten kasstroom in GBP uit hoofde van afgesloten, maar nog niet uitgeleverde verkooporders. De per periode te verwachten kasstroom in GBP uit hoofde van uitgeleverde verkooporders blijkt uit de debiteurenadministratie.


\stepcounter{artikel}
\subsection*{Article \theartikel \hspace{1em} Identification of hedging instruments}
\addcontentsline{toc}{subsection}{Article \theartikel \hspace{1em} Identification of hedging instruments}
De omvang en looptijd van de af te sluiten valutatermijncontracten worden bepaald overeenkomstig de omvang en verwachte betalingstermijnen van de
afgesloten verkooporders. Van de per periode afgesloten valutatermijncontracten worden de volgende gegevens vastgelegd in de administratie:

- afsluitdatum;

- tegenpartij en contractnummer;

- omvang in GBP;

- termijnkoers;

- omvang in EURO;

- datum afloop.


\stepcounter{artikel}
\subsection*{Article \theartikel \hspace{1em} Testing the effectiveness}
\addcontentsline{toc}{subsection}{Article \theartikel \hspace{1em} Testing the effectiveness}
De effectiviteit van de hedge zal aan het einde van iedere periode worden beoordeeld door het vergelijken van de kritische kenmerken van de lopende verkooporders en van de ingenomen valutatermijncontracten. Als er sprake is van een overhedge, worden eventuele verliezen als gevolg van deze overhedge direct ten laste van de winst-en-verliesrekening verwerkt.



\newpage
\section*{Administrative organization and internal control}
\stepcounter{artikel}
\subsection*{Article \theartikel \hspace{1em} Principles of the AO/IC}
\addcontentsline{toc}{subsection}{Article \theartikel \hspace{1em} Principles of the AO/IC}


\stepcounter{artikel}
\subsection*{Article \theartikel \hspace{1em} Responsabilities and authorizations}
\addcontentsline{toc}{subsection}{Article \theartikel \hspace{1em} Responsabilities and authorizations}

\begin{table}[!h]
    \centering
    \caption{Responsabilities and authorizations concerning treasury}
    \begin{tabular}{l l l}
        \toprule
        \textbf{Function} & \textbf{Responsabilities} & \textbf{Authorizations} \\
        \midrule
        The managing board & blabla & blabla \\
        Department managers & blabla & blabla \\
        Employees finance department & blabla & blabla \\
        \bottomrule
    \end{tabular}
    \label{tab:functiescheiding}
\end{table}


\stepcounter{artikel}
\subsection*{Article \theartikel \hspace{1em} Management information}
\addcontentsline{toc}{subsection}{Article \theartikel \hspace{1em} Management information}


\section*{Application of the treasury letter}
\stepcounter{artikel}
\subsection*{Article \theartikel \hspace{1em} Implementation}
\addcontentsline{toc}{subsection}{Article \theartikel \hspace{1em} Implementation}
Deze treasury letter treedt met terugwerkende kracht in werking vanaf 1 januari 2018 en vervangt voorgaande generieke documentatie op 13 november 2008 is vastgesteld door de Raad van de gemeente Ede.

%\printbibliography
%\addcontentsline{toc}{section}{Referenties}